\section{Method}
\label{sec:method}

In this section we will discuss some methodological concepts that we find especially interesting in relation to our project and which we think have had a direct impact on the end result.

\subsection{Problematisation}
As we started the project with DMA, the general idea was about implementing an e-learning platform to help facilitate the learning experience of its members. However, as we moved through the MUST method and as we finished the in-line analysis, we realised that this was not possible, due to the lack of genuine user participation\cite{bodker} and the limited amount of users, that were willing to participate in the project.

Once we analysed the Environment, Business and IT Strategies and the Work domains \cite{bodker}, it was clear that we would not be able to propose an innovative idea about a new e-learning platform but rather propose ideas based on the existing ones.

Through the Innovative Technology Analysis\cite{bodker}, we examined how new technological potentials can impact or alter the business strategy. This was done by performing interviews with the employees involved in the potential uses of IT. However, we were not able to get any documents concerning the existing IT system description or the one currently being developed and we were limited to performing a SWOT analysis.

As a result, the outcome of the in-line analysis phase, which also includes a problematisation of the premise for the entire design project, shifted towards a more technically general approach. This approach consisted of using the existing emailing platform of DMA, in order to attract more members into using the provided e-learning solution by DMA.


\subsection{Spokespersons}
In our project there has been quiet few spokespersons involved. This was caused by first of all our big focus on problematisation combined with the short timeframe in which this project was executed. The main spokespersons during our project were:

\begin{itemize}
\item Lene Rybner - DMA
\item Birgitte Rode Diness - Rigshospitalet
\item Claus Arwilk - DMA
\end{itemize}


Among these spokespersons only Lene was the only person participating in our steering committee and was a representative of DMA. Lene also had great responsibility in the project as she was partly responsible for helping us reaching out to possible spokespersons in the medical field and creating interessement, in order to lock these in, as discussed in\cite{callon}. Birgitte was representative of course creators in the medical field as she had created a blended learning course and had contact with DMA. Claus representative of the IT development team at DMA who would potentially be responsible of implementing the ideas.

During our project we had attempted to reach out to several doctors, who would be representative of different kinds of doctors. As we in our project greatly needed to create interessement and have people become spokespersons for the different roles. This became a real issue as we had to rely on Lene as our contact to DMA to help us with creating the interessement and contact to doctors who fit the different audiences such as course creators, course participants and so forth.

Ultimately, we were so time constrained that when we finally got contact to a group of people who had attended Birgittes blended course we felt that our only option in order to actually utilise this contact to hear and use their opinion was to send out a questionnaire to have them represent the role of course attendant, as discussed in the course participants section on page \pageref{indepth:questionnaire}.

We also spoke to another doctor, Jacob Melchiors, from Rigshospitalet, who would have been a great representative of the course participants. He would also have liked to assist us in our project and possibly participate in our steering committee. But this was rather late in our project and was mainly based upon his interest in learning methods and e-learning solutions. At that time we were already at a point, where we were just about to present our idea for our steering committee, Lene, and thus no longer considering designing a e-learning platform in reality, but instead move to the mandated idea of our project, where we would develop ideas to promote existing course material.

Having established contact this late meant that we had no time to meet with Jacob prior to obtaining the mandate for our direction, at which point we felt it would be pointless to include him in our steering committee as he would not have a lot of influence afterwards. Thus, in order to not waste his time without giving him influence we chose to stay with our current project construction. This was unfortunate as we had been looking for a representative such as Jacob, as he was able to represent the target audience which DMA had in mind in their case description.

The case with Jacob along with Birgittes course participators where we had to make an number of time sensitive decisions which likely greatly impact our project as discussed in \cite{key-to-success-p1}, where as in a utopian world we would have preferred to expand our steering committee in order to include representatives of each of the affected roles as suggested in \cite{callon}, which we somewhat lacked in our project. Thus having had a few course creators and course attendees, possibly in different age groups, as participants could possibly have changed the direction of our project entirely.

Accordingly more involvement or data from the IT development team might also have proven fertile, as we believe integration of any innovation is essential to its success in cases such as ours where users want to access the content we are trying to provide in a simple and streamlined way. Unfortunately this was not really possible given the issues described in the implementation section on page~\pageref{sec:implementation}. This might have been less of a problem if we had person from IT in our steering committee, such that this person could participate in the design of our innovation.
