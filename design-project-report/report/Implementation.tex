\section{Implementation strategy and plan}
\label{sec:implementation}
The IT systems and IT platform section above explains the situation to some extent, but there are some considerations that we will discuss further here.

DMA is currently in the process of designing their new website, and so the integration of our innovation into their new website needs to be considered. We have been in contact with Claus, the project leader on the new website project, and requested any available resources on the designs of the new website. However, they did not have any such resources available at the time, and so we have been forced to design against an unknown website design. We have therefore designed our mock-ups and innovation approach very generally, so that the “building blocks” we propose can be included in some form or shape regardless of the specifics of the website.

Because we do not have direct access to the new website, we run a risk of the website development team not being able to incorporate our innovation on their current schedule and budget. While not monumental in scope, our additions will need some extra working hours to be incorporated into the new website.

Our innovation of hooking and reeling (see page \ref{hookandreel}) can be coated in many different colors. The newsletter version is the very basic version, where a hook in the shape of a video, link to a test or similar is presented along with some very concise basic information about it. This is then simply added to the already existing newsletter. The course page version is the more “advanced” version which is a page on the website with all the same information as the newsletter version but with more information added, such as participant reviews, relevant courses and the likes. If this latter version is implemented, the newsletter version can also simply point to the course page where more information can be found as well as the actual hook.

Whichever solution is used opens up for a wide range of usage possibilities. Once the hook is created and presented in the form of a course page or just a snippet used in the newsletter, this resource can also be presented in any other relevant media. The hook (possibly with a link to the course page) might be presented on relevant Facebook pages, LinkedIn profiles, message boards or any other venue for sharing of information. Thus the innovation is very flexible and can easily be connected to changing circumstances.

\subsection{Technical}
Our solution poses no new technical requirements--every part of the innovation uses already existing solutions, be it the website, newsletter, facebook group, video sharing site (youtube), test hosting site (itslearning) and so on. The only technical needs is then that the upcoming website project gets extended slightly to support the course page and possibly also the course feed. DMA can get by without these by only presenting the hooks in the newsletter and copying the same presentation of the hook on other media, but a central place for these courses in the form of the course pages would boost its effectiveness. All other requirements are already posed by the upcoming website in order for it to be able to function as intended However, keeping our suggested innovation in mind when thinking of the new website could help clarify exactly what is required of the new website platform.

\subsection{Organisational}
The biggest organisational challenge will be deciding who will be responsible for creating, and actually creating, the course hooks. DMA and the course/learning responsible people there would initially do this, but course hooks created by the course source/creator could be preferred.

Once the course hook have been created it needs to be advertised. This requires the hook to be presented with the simple relevant information such as time requirements of the hook, type of the course etc. as described earlier. Once that information has been collected, the hook and information about it needs to be presented in the newspaper as well as other media by which DMA reaches out to the doctors. If the course pages and feeds are implemented, this hook also needs to be added to a course page as well as the website feed. This would be DMA’s responsibility as they are already advertising courses to the doctors through these very same media.
