\section{Objective}
\subsection{Objective and Premise of the Design Project }
The design project is based on the case as presented by DMA entitled “Digital One-Day Courses for Doctors” <Reference to the case description as an appendix?>. The direct goal of the case description is the development of a “digital learning platform”. The motivation is to provide to the more than 27000 members (per January 1. 2014) more courses on demand and give them an overview of available online courses both nationally and internationally. 

The DMA had a focus group develop requirements for the platform and easy access triumphed over variety, looks and features as the most important consideration. Additionally, the case asks us to consider the following:

\begin{enumerate}[A.]
\item Will there be in a foreseeable future sufficient demand for digital learning services and/or platform from DMA members?
\item What do the end users require from a digital learning platform? 
\item How can a digital learning platform be implemented within the existing organisation of DMA?
\item What is the ‘minimal viable product’ i.e. the most simple inexpensive product that allows DMA to enter this market?
\end{enumerate}

And in the pursuit of these to explore:

\begin{itemize}
\item What are the reasons for the current lack of incitement for the digital offers?
\item How can digital learning  be embedded in the daily work of doctors, who typically demand a decent work-life balance, and who typically prefer education to be something that happens in a formal setting such as a course or conference?
\item Is it more suitable continuing the current ad-hoc strategy, or does the increased uptake of digital learning offer a more holistic, top-down strategy and thus also a platform?
\item What devices will best support the learning situation: Computers, tablets or smart phones, or should the solution be compatible with all types of devices?
\end{itemize}

DMAs expectations to the solution were:

\begin{itemize}
\item A specification of a strategy that will provide the members of the DMA with more digital learning possibilities. This includes considerations about the cost structure, resource requirements etc.
\item A specification of the user requirements and design of a digital learning service offers that sufficiently fit with the existing organisation in DMA.
\item A design proposal for the learning service, including considerations regarding distribution of this.
\item A specification of the overall technical requirements for such a solution, including recommendations for the software platform, and considerations regarding integration with the existing web page www.laeger.dk.
\item Considerations regarding channels for distribution of the service (e.g. via the existing homepage or via app stores).
\end{itemize}

We had some ideas to what might fulfil the noted requirements as we started the project. The main idea we discussed was an e-learning platform where courses were “chopped up” into 5 minute bits that would be easy for the doctors to fit in and follow in their busy schedules. <Don’t know where else to fit this in?>\footnote{TODO: FIX THIS}

In the initial phase of the MUST method, our approach was to establish a thorough understanding of the organisational setup of DMA and identify the main stakeholders. This allowed us to understand the scope of the project and eliminate all potential misunderstandings between the project group and the steering committee. In the first meeting with the steering committee , which consisted of Lene Rybner as the sole member, we got to know the project better and what the wishes and expectations of DMA were.

Based on our understanding and expectations at the time, we made a baseline plan for the project <Figure reference, insert figure below>\footnote{TODO: FIX THIS}. The deadlines and outcome of each phase of the project was dictated by the course structure, so this was unchanged, but we later made an updated baseline plan <Figure reference>\footnote{TODO: FIX THIS} that describes the actual activities in each phase. 


\subsection{In-line Analysis Phase}
\subsubsection{Approach}
Our approach in the in-line analysis phase was to, firstly, figure out the major work domains to get a sense of where the IT system was used and where innovation could actually be required, and secondly, dig a little deeper into DMA’s organization by discovering some of their business strategies behind course creation, how are they created, who provides them and what are the main platforms of the courses.

\subsubsection{Environment}

The Danish Medical Association (DMA), Lægeforeningen in Danish, is an association that aims to represent all Danish doctors and their associated organisations such as unions, tying them all together and providing benefits for all its members. There are three major medical organisations under the DMA. The structure is illustrated in figure \ref{dmaorganisation}\footnote{TODO: FIX THIS}.

The first is “Yngre Læger” (YL), an association for younger doctors with close ties to doctors advancing their studies in the “klinisk basisuddannelse” (KBU), those taking their introduction to their speciality and those who are actively specialising.

The second is “Forening af Speciallæger” (FAS) which is a union for consulting doctors that have studied a speciality.

Lastly is “Praktiserende Lægers Organisation” (PLO) which is an organisation for the practicing family doctors, who usually run their own local clinic for the nearby population.

Danish doctors are generally associated with one of these three organisations and are thus members of the DMA, having access to the DMA provided courses and other benefits.

\begin{figure}
\label{dmaorganisation}\footnote{TODO: FIX THIS}
\end{figure}

\subsubsection{Potential and Needs}
DMA members need to take courses in order to stay up to date with the latest information about medicine practices. So far the existing e-learning platforms that provide the necessary courses are not widely used by the doctors, in some cases this is because they consider it to be time consuming. Some doctors also consider the contents of the e-learning courses to be too easy and superficial <Reference?>. What is needed is a new way to deliver information to the doctors, such that we can engage them in a way that eliminates the concerns of the current e-learning solutions.


\subsubsection{Requirements and Conditions}
DMA members are looking for a new e-learning platform with a different approach compared to the existing offers provided by DMA, such as BMJ Learning, since the DMA members do not currently utilise the offers as much as expected. DMA would like to have the content focused on including the entire spectrum (known as the 7 roles of physicians) of knowledge areas, which a modern doctor should be educated in, for example there should be topics dealing with communication, administration and so on. Currently surveys performed by DMA show that some doctors prefer to strictly focus their education on medical expertise and entirely neglecting being educated in the remaining 6 roles.

\subsubsection{Business Strategy} 
\begin{description}
\item[Goals:] We aim to find an approach for busy doctors to learn course material that will work for them. The approach should be realistic and feasible within the constraints of the target association DMA, and should actually engage the doctors and fit their needs such that the doctors will use it in practice.
\item[Business Processes:] We did not consider business processes very relevant in this phase of the project. DMA has a dominant market position with 98\% of Danish doctors being members and a fixed membership fee subscription model, which was something we learned later.

We do not consider business processes relevant at this time, as it is outside the scope of our study. 
\item[Challenges and Problems:] There are various challenges associated with solving the problem of doctors not actively continuing their education despite attempts to facilitate their learning. Firstly, we had to establish what was actually causing the problem. The cause seems to be that the doctors do not have time in their busy schedules to take hours or entire days off to study and take courses, but is this actually the main cause? Is there more to it? Are other factors in the way of doctors keeping up with their fields?

If the stated problem is confirmed to be what it seems to be, then we still have the problem of figuring out an alternative method of providing the course materials in a manner that is accessible to the doctors. And even with accessible learning, the approach still needs to motivate the doctors to actually use the solution. So we need to understand exactly what the doctors are looking for in a way of consuming courses and what exactly will suit their needs.
\end{description}
\begin{figure}
Business Model Canvas v1\footnote{TODO: FIX THIS}
\end{figure}

\subsubsection{Work Domains}
We have identified two primary work domains: Course creation and Doctors practice. 


