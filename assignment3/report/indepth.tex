\section{Methodical approach}
In this phase we interviewed a course creator who was also a doctor, sent out questionnaires to a range of doctors and interviewed a key person, Lene Rybner, from DMA. We do not have the answers to the questionnaires in time for this report, and we would also have liked to do some observation on especially doctors work practices, to get a more nuanced view of their schedule and work day, but this has not been possible.\\
We have also investigated existing types of course teasers that would be relevant to our focus. This will be described more in the following.

\section{In-depth analysis report}

\subsection{Backdrop and focus}
Interviewing and questioning doctors and course creators helped us focus our project. The plan for this phase was to investigate the doctors and course creators work practices, but in doing so, we discovered that the work practices at DMA regarding course promotion was the crucial point that tied the doctors and courses together.\\
This lead us to slightly change our focus from fitting courses into the doctors busy schedule, to bringing existing courses to the doctors attention within the strict time constraints they work with.
The reason for this change is, that we keep discovering new sources of medical courses that could be relevant to the doctors. There is no shortage of courses out there, so instead we focus on DMA’s unique role as an organization. When a course is endorsed by DMA, the doctors are more likely to look at it. The challenge then is to catch the doctors interest and allow them to investigate these courses.

\subsection{Significant work practice characteristics}
From an interview with a person who is both a doctor and a course creator of CPD material the project group was able to identify work practices at both ends of the table and help view the project from two different point of views.

We started asking her questions with regards to her role as an crouse creator. She had helped creating and managing a course on genetics which ran as an introductory course to raise awareness of genetics being a fully fledged medical field. The online platform chosen for the course was itslearning where all course material, exams and discussions were kept. This course had previously failed to start twice due to lack of participants. Before the information on the course had been posted to a system that required a login and it wasn’t until they used the DMA’s monthly email newsletter that they got enough participants.

We then switched question topics and started asking her questions with regards to her role as an doctor. When asked if she would have 5 or 10 minutes spare time at some point in her schedule, could she see herself taking a fragment of an e-learning course, her answer is that that the material would have to be very accessible. She has signed up for BMJ before but the slightest resistance such as a complicated login to a different system kills the “will to learn” moment.

The email newsletter is therefore a critical channel in engaging doctors and it makes good sense based on their current work practices. The email doesn’t require a complicated login and it’s right there in front of them when the doctor uses his/hers computer. Doctor's are usually too busy with other things on their mind than logins to different systems so the accessibility to the course material is a critical point as well.

Throughout our interviews with Lene, we discovered that she is the one responsible for making the available courses or material known to the doctors' community. She has created various accounts on social networks (Facebook, LinkedIn), through which she promotes various courses that are available on DMA, BMJ, MOOCs and other e-learning platforms. In our last meeting we discussed about what would be the preferable time limit for an online course, we suggested a 7 minute time limit  and she replied that it is too much for the doctors to spend that amount of time.

In order to suggest various courses to the doctors she has to search through the courses herself in the above mentioned sites and post them on Facebook groups and various home pages that doctors visit. Lately she has started posting pictures concerning medical problems that are not “graphic” as she stated, but interesting enough to make the doctors look up the courses she suggests. However, she has no way of knowing if her methods work, because everything is hosted in websites that are not hosted by DMA (Facebook, LinkedIn), hence she cannot have any information about the traffic.

In the next section, we will talk about our ideas of solutions to this discovery.

\subsection{Goals, problems, needs, and ideas for solutions}
The solution we propose is influenced by our newly focused scope: the doctor's time. Already there are a lot of courses available for doctors to choose from whether it’s from DMA, BMJ or MOOCs. As mentioned before, the project group’s assumption is that the doctors are too busy to find out themselves which courses are relevant to their CPD. The problem is not that doctors don’t want to take the courses or use e-learning platforms, but more of that they aren’t getting properly introduced to available courses. We propose creating a well-defined framework for the DMA to work within to help them to sell the courses by using something we call a course hook.

In the current in-depth analysis phase we discovered that the DMA’s newsletter was important. Based on our focused scope on the doctor's time, one solution would be that the DMA itself would pick out a subset of available courses from the vast variety of material and advertise them in the newsletter based on medicine specialities. By using our terms before, we can give an example:

\begin{description}
    \item[\textbf{Headline:}] "How well do you know Genetics?"
    \item[\textbf{The hook:}] [VIDEO] A 5 minute introduction to the upcoming course in Genetics
\end{description}

In this example, the DMA has handpicked a course on Genetics which they feel is a very appropriate course for genetics. There is a number of important things in this example. First, the advertisement is selling a genetics course and does so by challenging the doctors knowledge. Second, the doctor can see exactly what kind of introduction this is and how long it takes (the hook). The doctors are therefore able to make a decision based on their strict schedule whether or not they have a spare time.

\subsection{Suggested priorities}
The priorities that we suggest based on the above is that DMA should make these “hooks” so that they can improve their newsletter, in a way that the doctors will be more informed about the courses that they are about to take and their timeframe. In addition, DMA has to make sure that they strictly comply to the time limits that are presented at the newsletter, so that the “will to learn” will not be reduced by such trivial obstacles.

In order for DMA to create the above mentioned “hooks”, a framework needs to be developed that will create these short sessions of e-learning and distinguish the various courses that want to promote through DMA’s newsletter.

\section{Scope for upcoming phase (Innovation)}
In our upcoming innovation phase we investigate exactly how a solution to the problem narrowed down in this in-depth phase might be found, and just how this solution can realistically be implemented in practice.

Before the solution is implemented, we need to have established a coherent vision between ourselves and our stakeholders so that we all agree on how to proceed. To achieve this, we need to design potential mock-ups demonstrating our vision of how to advertise a course. This includes designing and incorporating such a mock-up into their newsletter, as well as sketching out how the course advertisements might be featured in other media such as their website, facebook group, linkedin profile and other assets by which we can reach the doctors.

An important part of presenting these mock-ups is to describe and clarify the intent behind them, as well as conveying the bullet point “hooks” of what makes these introductions captivating that should be kept in mind. This includes an emphasis on the “hook” needing to be short or satisfy exact time specifications. Other materials to draw from in creating these hooks include an actual introduction video from BMJ\footnote{https://www.youtube.com/watch?t=7\&v=e1HrFp6Xv7w} and short tests such as on the itslearning Genetic’s Course introductory test or one from BMJ\footnote{https://www.facebook.com/BMJLearning/photos/a.248433962010565.1073741832.134477093406253/440926912761268/?type=3\&theater}. Other ideas to look into are a market study of how course creators advertise their courses. An example is BMJ which has a ratings system, reviews and suggestions such as “Recommended for your profession”, “Most popular” and the likes. These would be good to draw inspiration from, as would how PLO advertise their courses and how courses are advertised on the facebook group for course sharing used by doctors.

As a part of implementing this coherent vision we need to identify which qualifications are needed to follow our vision of change. Relying on our anchoring, we need to establish what is required of each of our stakeholders in order to realize the solution. For example, the newsletter writers need to be informed on how to catch the interest of doctors, the content pushers need to ensure that the introductory videos are provided on an easy-access platform such as YouTube or Vimeo and so on.

Lastly, once everything is implemented, our plan should include steps on how to further improve the solution. This includes studying the effects of these “hooks” to see what works and what does not, such as by asking the stakeholders directly and get testimonials from the users.

There are also a few risks in promoting the courses that need to be addressed. We need the course creators to generate the hooks to captivate people to attend the courses, but how willing can they be expected to be to do this additional work? Quality of the hook is essential for it to be effective, and so we need to motivate the course creator by pointing out advantages to them in creating an excellent hook.


