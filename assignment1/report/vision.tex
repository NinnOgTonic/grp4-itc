\section{Vision of the Project}

\subsection{Problem}
The Danish Medical Association (DMA) currently has a large amount of online courses for doctors in their system, both free of charge and paid for. The courses’ topics are also widely ranged from informing doctors of new medical treatments to development on their own personal skills. In spite of this, DMA sees a very little usage of its digital learning platform, especially from the older generation of doctors where the feedback indicates that it is a “break from responsibility” to use the system whilst they are working. This indicates that time does play a large factor here, since the courses require an allocated time in the doctor’s daily routine. Having finished a course, the doctor can add the newly acquired skills to his/her CV but not much more, which does raise the concern that the motivation for taking the courses might be little.

The DMA has a lot of data from its website which indicates that doctors do visit the learning part of the website, but generally spend very little time there. On average they spend a few minutes, which is far less than the necessary time required to complete a course.

Additionally the traffic to the DMA website mainly comes from desktop computers, but traffic from mobile devices is increasing. The current learning platform is not optimised for mobile devices.
Lastly even though there might be allocated time in doctors work schedules, there is no real incentive, and keeping up with new knowledge might just in the end become a hassle as taking away precious office hours which some doctors think is better spent practicing medicine.

\subsection{Background}
It is important for any professional to continue developing his/her skillset, even more so when it comes to medical personnel, education can often impact a patient’s life greatly. In some cases, education becomes the difference between life and death. For this exact reason, learning and improving the doctors’ skillset is an essential part of the danish doctors’ oath.

Currently danish doctors’ preferred way of obtaining new knowledge is by attending conferences and lectures. This comes in contrast to how doctors in other countries obtain new knowledge. For example, in Great Britain they have to obtain a certain amount of education points each year, which can be obtained through online courses. This has lead to the success of e-learning solutions such as BMJ Learning.

The DMA also wishes to prepare its learning platform for a new generation of doctors, who it believes might be more willing and prepared to use new devices and platforms to acquire new knowledge.

\subsection{Vision}
Our design project will focus on this problem by identifying the requirements needed for a successful online learning system for busy doctors, what platforms the service should be provided on and study the demand for this kind of system with regards to the new generation of doctors who are more likely to use new technology in their studies. This includes investigating the possible and optimal structure of the courses that DMA provides, so that more doctors actually take them.

The ideal outcome could be that the system will be widely used by DMA so that doctors can learn new things or refresh their knowledge. While there are courses already available for the doctors, they are not really accessible due to time constraints. This is the problem we aim to solve.

\subsection{Next Steps}
Our next steps will be to visit the DMA and discuss with the project responsible about the tools or the access that we are going to need to continue our work. Furthermore, we will need to identify the groups of people (stakeholders) with whom we are going to discuss our project. Finally, there has to be done some “field” work, meaning that we will possibly talk to some of the creators of the online courses or even have some workshops with them. From there on, we will narrow down the problem and pursue possible solutions.

